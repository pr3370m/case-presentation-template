\documentclass[10pt]{beamer}

%================
% 	PREAMBLE	
%================
\usetheme[progressbar=frametitle]{metropolis}
\usepackage{appendixnumberbeamer}
\usepackage{booktabs}
\usepackage[scale=2]{ccicons}
\usepackage{pgfplots}
\usepgfplotslibrary{dateplot}
\usepackage{xspace}
\newcommand{\themename}{\textbf{\textsc{metropolis}}\xspace}

% Customization - Pritom
\usepackage[super]{nth}
\usepackage[version=4]{mhchem}
\usepackage{textcomp}
\usepackage{xcolor}
\definecolor{green}{HTML}{006600}
\definecolor{lgreen}{HTML}{006600}
\definecolor{red}{HTML}{ee0000}
\definecolor{iblue}{HTML}{34484c}
\usepackage[export]{adjustbox}
 
\usepackage{graphbox}
%\setbeamertemplate{footline}{%
% 	\hspace{2mm}\includegraphics[align=c, height=0.3cm]{icddrb-grey-logo-900p} \hspace{0.5mm}
% 	icddr,b: solving public health problems through innovative scientific research \hfill \insertframenumber\,/\,\inserttotalframenumber\kern1em%
%% 	\hfill%
% 	\usebeamercolor[fg]{page number in head/foot}%
% 	\usebeamerfont{page number in head/foot}%
%%	\hfill \insertframenumber\,/\,\inserttotalframenumber\kern1em%
%}

\setbeamercolor{footlinecolor}{fg=white,bg=iblue}
\setbeamertemplate{footline}{%
\begin{beamercolorbox}[sep=1em,wd=\paperwidth,leftskip=0.2cm,rightskip=0.5cm]{footlinecolor}
%	\hfill
	\includegraphics[align=l, scale=0.10,height=10pt]{icddrb-word-logo-w} 
	\hspace{0.3cm}%
%	\scriptsize{}
%	\hfill \insertframenumber\,/\,\inserttotalframenumber
%	\includegraphics[scale=0.35,height=12pt]{icddrb-grey-logo}
\end{beamercolorbox}%
}


%================
% 	BODY
%================
\title{Case Presentation}
%\subtitle{A modern beamer theme}
% \date{\today}
\date{}
\author{Dr. Visnu Pritom Chowdhury}
\institute{Clinical fellow, NCSD, icddr,b}
\titlegraphic{\hfill\includegraphics[height=1.0cm]{icddrb-word-logo}}
%\logo{\includegraphics[height=0.65cm]{icddrb-grey-logo-custom.png}}


\begin{document}

\maketitle

%\begin{frame}{Table of contents}
%  \setbeamertemplate{section in toc}[sections numbered]
%  \tableofcontents[hideallsubsections]
%\end{frame}

%\section{History}

%\begin{frame}[fragile]{Patient particulars}
%
%  The \themename theme is a Beamer theme with minimal visual noise
%  inspired by the \href{https://github.com/hsrmbeamertheme/hsrmbeamertheme}{\textsc{hsrm} Beamer
%  Theme} by Benjamin Weiss.
%
%  Enable the theme by loading
%
%  \begin{verbatim}    \documentclass{beamer}
%    \usetheme{metropolis}\end{verbatim}
%
%  Note, that you have to have Mozilla's \emph{Fira Sans} font and XeTeX
%  installed to enjoy this wonderful typography.
%\end{frame}
%\begin{frame}[fragile]{Sections}
%  Sections group slides of the same topic
%
%  \begin{verbatim}    \section{Elements}\end{verbatim}
%
%  for which \themename provides a nice progress indicator \ldots
%\end{frame}

%\section{Titleformats}

{\setbeamertemplate{frame footer}{icddr,b}
\begin{frame}{Patient particulars}
%	\themename supports 4 different titleformats:
	\begin{itemize}
		\item Name: %Jannat Bibi
%		\item \textsc{Smallcaps}
%		\item \textsc{allsmallcaps}
%		\item ALLCAPS
		\item Age: 9 years 
		\item Sex: Female
		\item Address: Gopaldi, Araihazar, Narayangonj
		\item Admission: \nth{17} June, 2018 at 8:31 PM
	\end{itemize}
%	They can either be set at once for every title type or individually.
\end{frame}
}

%{\setbeamertemplate{frame footer}{icddr,b}
%\begin{frame}{Using Columns}
%	\begin{columns}
%		\column{0.5\textwidth}
%		\begin{itemize}
%			\item Name: %Jannat Bibi
%			\item Age: 9 years 
%			\item Sex: Female
%			\item Address: Gopaldi, Araihazar, Narayangonj
%			\item Visit date: \nth{17} June, 2018
%		\end{itemize}
%		\column{0.5\textwidth}
%		\begin{figure}
%			\includegraphics[scale=0.115]{sub}
%%			\caption{The 9 years old patient}
%		\end{figure}
%	\end{columns}	
%\end{frame}
%}

\section{at Short Stay Unit}

{\setbeamertemplate{frame footer}{icddr,b}
\begin{frame}
	\frametitle{Subject}
	\begin{figure}
%		\includegraphics[scale=0.15,fbox]{sub}
%		\includegraphics[scale=0.15,frame]{sub}
		\includegraphics[scale=0.14]{scientist-t}
		\caption{The 9 years old patient at emergency ward.}
	\end{figure}
%	text
\end{frame}
}

{\setbeamertemplate{frame footer}{icddr,b}
\begin{frame}{Presentation at SSU}
	\begin{itemize}
		\item Condition: Lethargic 
		\item Eyes: Sunken
		\item Mucosa: Dry
		\item Thirst: Unable to drink 
		\item Skin turgor: Reduced
		\item Radial pulse: Absent 
		\item Dehydration: Severe
	\end{itemize}
\end{frame}	
}

{\setbeamertemplate{frame footer}{icddr,b}
	\begin{frame}{Management received}
		\begin{itemize}
			\item IV fluid (Acetate 3L) and G-ORS
			\item Azithromycin (STAT), suspecting Cholera
		\end{itemize}
	\end{frame}	
}

{\setbeamertemplate{frame footer}{icddr,b}
	\begin{frame}{Transfer details: SSU $\Rightarrow$ LSU}
		\begin{itemize}
			\item \textbf{Time}:\\
			{\color{blue}$\circ$} \nth{19} June, 2018 at 4:39 PM
			\item \textbf{Cause}: \\
			{\color{blue}$\circ$} Repeated dehydration \\
			{\color{blue}$\circ$} Fever \\
			{\color{blue}$\circ$} Acute watery diarrhea 
		\end{itemize}
	\end{frame}	
}

\section{at Longer Stay Unit}

{\setbeamertemplate{frame footer}{icddr,b}
\begin{frame}{Presenting complaints}
	\begin{itemize}
		\item Watery stool with vomiting for 3 days
		\item Fever for 2 days
		\item Burning sensation during micturition with lower abdominal pain for 2 days
	\end{itemize}
\end{frame}	
}

{\setbeamertemplate{frame footer}{icddr,b}
\begin{frame}{Present illness history}
	\begin{itemize}
		\item Reasonably well 3 days back
		\item Acute onset, passage of watery stool 3 days prior to admission with no associated blood or mucus or straining
		\item Fever for last 2 days
		\item Dysuria and abdominal pain for 2 days
	\end{itemize}
\end{frame}
}

{\setbeamertemplate{frame footer}{icddr,b}
	\begin{frame}{Past illness history}
		\begin{itemize}
			\item No significant history of past illness
		\end{itemize}
	\end{frame}
}

{\setbeamertemplate{frame footer}{icddr,b}
\begin{frame}{Socio-economic history}
	\begin{itemize}
		\item Tin shed housing 
		\item Supply water
		\item Father’s occupation: Rickshaw puller
	\end{itemize}
\end{frame}
}

{\setbeamertemplate{frame footer}{icddr,b}
	\begin{frame}{Drug and allergy history}
		\begin{itemize}
			\item No prescribed drug history
			\item No drug allergy was found
		\end{itemize}
	\end{frame}
}

%\section{Examination}

{\setbeamertemplate{frame footer}{icddr,b}
\begin{frame}{General examination}
	\begin{itemize}
		\item Weight: 18.10 kg
		\item Height: 
		\item Weight for age z-score: -3.02 SD (WHO AnthroPlus)
		\item Temperature: 36.6\textdegree{}C
		\item Pulse (Radial): 110 /min (Regular)
		\item Respiratory: 24 /min
		\item Sp\ce{O2}: 99\% without \ce{O2} support
	\end{itemize}
\end{frame}
}

{\setbeamertemplate{frame footer}{icddr,b}
	\begin{frame}{General examination}
		\begin{itemize}
%			\item Condition: Lethargic 
			\item Eyes: Sunken 
			\item Mucosa: Dry 
			\item Thirsty
			\item Skin turgor: Reduced
			\item Dehydration: Some 
		\end{itemize}
	\end{frame}
}

{\setbeamertemplate{frame footer}{icddr,b}
	\begin{frame}{General examination}
		\begin{itemize}
			\item Jaundice: no
			\item Cyanosis: no
			\item Edema: no
			\item Lymphadenopathy: no
			\item HEENT: NAD
		\end{itemize}
	\end{frame}
}

{\setbeamertemplate{frame footer}{icddr,b}
\begin{frame}{Systemic exam: Cadiovascular}
	\begin{itemize}
		\item Apex beat: Normal
		\item Heart sounds: S\textsubscript{1} + S\textsubscript{2} + 0
		\item Murmurs: No
	\end{itemize}
\end{frame}
}

{\setbeamertemplate{frame footer}{icddr,b}
\begin{frame}{Systemic exam: Respiratory}
	\begin{itemize}
		\item Chest movement: Symmetrical 
		\item Intercostal recession: No, emaciated
		\item Tracheal tug: No
		\item Airway: Clear
		\item Lungs: Vesicular breath sounds, no added sounds
	\end{itemize}
\end{frame}
}

{\setbeamertemplate{frame footer}{icddr,b}
\begin{frame}{Systemic exam: Abdomen}
	\begin{itemize}
		\item Contour: Scaphoid
		\item Distension: No
		\item Tenderness: Lower abdominal
		\item Rigidity: No
		\item Organomegaly: No
		\item Bowel sound: Present
	\end{itemize}
\end{frame}
}

{\setbeamertemplate{frame footer}{icddr,b}
\begin{frame}{Systemic exam: Nervous}
	\begin{itemize}
		\item No abnormality detected
%		\item Orientation: good
%		\item Co-operative ability: NAD
%		\item No signs of meningeal irritation
	\end{itemize}
\end{frame}
}

{\setbeamertemplate{frame footer}{icddr,b}
\begin{frame}{Provisional diagnosis}
	\begin{itemize}
		\item Acute watery diarrhea (AWD) with, \\
		Some dehydration
		\item Probable urinary tract infection (UTI)
	\end{itemize}
\end{frame}
}

{\setbeamertemplate{frame footer}{icddr,b}
	\begin{frame}{Differential diagnosis}
		\begin{itemize}
			\item Watery diarrhea, due to: \\
			{\color{red}$\circ$} Cholera \\
			{\color{red}$\circ$} ETEC
			\item Fever, due to: \\
			{\color{red}$\circ$} Viral \\
			{\color{red}$\circ$} UTI 
			\item Lethargy, due to:\\
			{\color{red}$\circ$} Dehydration \\
			{\color{red}$\circ$} Electrolytes imbalance
			\item Dysuria, due to: \\
			{\color{red}$\circ$} UTI
		\end{itemize}
	\end{frame}	
}

%\section{Managements}

{\setbeamertemplate{frame footer}{icddr,b}
\begin{frame}{Management: Fluid}
	\begin{itemize}
%		\item \textbf{In SSU}: \\
%		{\color{lgreen}$\triangleright$} IV fluid (Acetate 3L) and G-ORS
		\item G-ORS, for some dehydration and
		\item Fluid ration for 6 hours (NS + 5\% Dextrose 0.5L), \\
		due to persistent vomiting
%		\textbf{In LSU}: \\ 
%		{\color{lgreen}$\triangleright$} G-ORS, for some dehydration and \\
%		{\color{lgreen}$\triangleright$} Fluid ration for 6 hours (NS + 5\% Dextrose 0.5L), \\
%		due to persistent vomiting
	\end{itemize}
\end{frame}
}

{\setbeamertemplate{frame footer}{icddr,b}
\begin{frame}{Management: Drugs}
	\begin{itemize}
%		\item \textbf{In SSU}: \\ 
%		{\color{lgreen}$\triangleright$} Azithromycin (STAT), suspecting Cholera
		\item Syr. Levofloxacin (UTI)
		\item Syr. Ranitidine
		\item Inj. Granesetrone
		\item Syr. Potassium chloride
%		\textbf{In LSU}: \\
%		{\color{lgreen}$\triangleright$} Levofloxacin (UTI)
	\end{itemize}
\end{frame}
}

%{\setbeamertemplate{frame footer}{icddr,b}
%\begin{frame}{Management: Other drugs}
%	\begin{itemize}
%		\item \textbf{In LSU}: \\ 
%		{\color{lgreen}$\triangleright$} Syr. Ranitidine \\
%		{\color{lgreen}$\triangleright$} Inj. Granesetrone \\
%		{\color{lgreen}$\triangleright$} Syr. Potassium chloride
%	\end{itemize}
%\end{frame}
%}

{\setbeamertemplate{frame footer}{icddr,b}
\begin{frame}{Management: Dietary}
	\begin{itemize}
		\item Full strength rice-suji 150 ml 2 hourly\\
		{\color{lgreen}$\diamond$} Volume: 99.5 ml/kg/day\\
		{\color{lgreen}$\diamond$} Calorie: 70 kcal/kg/day
	\end{itemize}
\end{frame}
}

%\section{Investigations}

{\setbeamertemplate{frame footer}{icddr,b}
\begin{frame}{Investigations: Microscopy}
  \begin{table}
  	\caption{Urine RME}
  	\begin{tabular}{r|c}
  		\toprule
  		pH 					& 6.5\\
  		Protein 			& \textcolor{red}{1$+$}\\
  		Ketones				& 2$+$\\
  		RBC (/HPF) 			& \textcolor{red}{1 -- 3}\\
  		Pus cells (/HPF) 	& \textcolor{red}{25 -- 30}\\
  		Epithelial cells (/HPF)	& 1 -- 3\\
  		Granular casts (/LPF) 	& 5 -- 9\\
%  		\midrule
  		\bottomrule
  	\end{tabular}
  \end{table}
\end{frame}
}

{\setbeamertemplate{frame footer}{icddr,b}
\begin{frame}{Investigations: Microbiology}
  \begin{table}
  	\caption{Stool CS}
  	\begin{tabular}{r l}
  		\toprule
  		Organism: & \textit{Aeromonas spp.}\\
  		\midrule
%  		Sensitivity pattern & \\
  		Tetracycline & \textcolor{green}{S} \\
  		Cotrimoxazole & \textcolor{red}{R} \\
  		Erythromycin & \textcolor{red}{R}\\
  		Ciprofloxacin & \textcolor{green}{S} \\
  		Azithromycin & \textcolor{red}{R} \\
  		Doxycycline & \textcolor{green}{S} \\
  		\bottomrule
  	\end{tabular}
  \end{table}
\end{frame}
}

{\setbeamertemplate{frame footer}{icddr,b}
\begin{frame}{Investigations: Microbiology}
  \begin{table}
  	\caption{Urine CS}
  	\begin{tabular}{c}
  		\toprule
  		No organism isolated\\
%  		\midrule
  		\bottomrule
  	\end{tabular}
  \end{table}
\end{frame}
}

{\setbeamertemplate{frame footer}{icddr,b}
\begin{frame}{Investigations: Biochemistry}
  \begin{table}
  	\caption{S. Electrolytes}
  	\begin{tabular}{r|l}
  		\toprule
  		\ce{Na+} & 129.3\\
  		\ce{K+} & \textcolor{red}{3.05}\\
  		\ce{Cl-} & 103.6\\
  		T\ce{CO2} & 12.5\\
  		Anion gap & 16.25\\
  		\bottomrule
  	\end{tabular}
  \end{table}
\end{frame}
}

{\setbeamertemplate{frame footer}{icddr,b}
\begin{frame}{Investigations: Hematology}
  \begin{table}
  	\caption{CBC}
  	\begin{tabular}{r|l}
  		\toprule
  		Hemoglobin & 15.2 gm/dl\\
  		Hematocrit/PCV & 42.7\%\\
  		\midrule
  		Total leucocyte count & 13.98$\times$10\textsuperscript{9}/L\\
  		\midrule
  		Neutrophil & 71.4\%\\
  		Lymphocyte & 17.6\%\\
  		Monocyte & 4.4\%\\
  		Eosinophil & 6.1\%\\
  		Basophil & 0.5\%\\
  		\bottomrule
  	\end{tabular}
  \end{table}
\end{frame}
}

%\section{Diagnosis}

{\setbeamertemplate{frame footer}{icddr,b}
\begin{frame}{Diagnosis}
	\begin{itemize}
		\item Acute watery diarrhea
		\item Urinary tract infection
		\item Electrolyte imbalance ($\downarrow$\ce{Na+}, $\downarrow$\ce{K+})
	\end{itemize}
\end{frame}
}

%\section{Outcome}

{\setbeamertemplate{frame footer}{icddr,b}
\begin{frame}{Outcome}
	${\color{green}\triangleright}$ Diarrhea resolved\\
	\begin{columns}[T,onlytextwidth]
		\column{1.0\textwidth}
		\metroset{block=fill}
		\begin{block}{Advices}
			Please continue medication regularly as prescribed, \\
			Please use clean and boiled water, \\
			Please maintain personal hygiene, \\
			Please contact with nearby physician/hospital if any problem arises.
		\end{block}
	\end{columns}
	${\color{green}\triangleright}$ Discharged with advice, \\on \nth{21} June, 2018 at 10:15 AM
\end{frame}
}


{%
\setbeamertemplate{frame footer}{icddr,b: solving public health problems through innovative scientific research}
\begin{frame}{}
	\centering \Huge
	\textbf{Thank you}
\end{frame}
}

%\begin{frame}{}
%	\begin{itemize}
%		\item 
%	\end{itemize}
%\end{frame}


%{
%    \metroset{titleformat frame=smallcaps}
%\begin{frame}{Small caps}
%	This frame uses the \texttt{smallcaps} titleformat.
%
%	\begin{alertblock}{Potential Problems}
%		Be aware, that not every font supports small caps. If for example you typeset your presentation with pdfTeX and the Computer Modern Sans Serif font, every text in smallcaps will be typeset with the Computer Modern Serif font instead.
%	\end{alertblock}
%\end{frame}
%}
%
%{
%\metroset{titleformat frame=allsmallcaps}
%\begin{frame}{All small caps}
%	This frame uses the \texttt{allsmallcaps} titleformat.
%
%	\begin{alertblock}{Potential problems}
%		As this titleformat also uses smallcaps you face the same problems as with the \texttt{smallcaps} titleformat. Additionally this format can cause some other problems. Please refer to the documentation if you consider using it.
%
%		As a rule of thumb: Just use it for plaintext-only titles.
%	\end{alertblock}
%\end{frame}
%}
%
%{
%\metroset{titleformat frame=allcaps}
%\begin{frame}{All caps}
%	This frame uses the \texttt{allcaps} titleformat.
%
%	\begin{alertblock}{Potential Problems}
%		This titleformat is not as problematic as the \texttt{allsmallcaps} format, but basically suffers from the same deficiencies. So please have a look at the documentation if you want to use it.
%	\end{alertblock}
%\end{frame}
%}

%\section{Elements}
%
%\begin{frame}[fragile]{Typography}
%      \begin{verbatim}The theme provides sensible defaults to
%\emph{emphasize} text, \alert{accent} parts
%or show \textbf{bold} results.\end{verbatim}
%
%  \begin{center}becomes\end{center}
%
%  The theme provides sensible defaults to \emph{emphasize} text,
%  \alert{accent} parts or show \textbf{bold} results.
%\end{frame}

%\begin{frame}{Font feature test}
%  \begin{itemize}
%    \item Regular
%    \item \textit{Italic}
%    \item \textsc{SmallCaps}
%    \item \textbf{Bold}
%    \item \textbf{\textit{Bold Italic}}
%    \item \textbf{\textsc{Bold SmallCaps}}
%    \item \texttt{Monospace}
%    \item \texttt{\textit{Monospace Italic}}
%    \item \texttt{\textbf{Monospace Bold}}
%    \item \texttt{\textbf{\textit{Monospace Bold Italic}}}
%  \end{itemize}
%\end{frame}

%\begin{frame}{Lists}
%  \begin{columns}[T,onlytextwidth]
%    \column{0.33\textwidth}
%      Items
%      \begin{itemize}
%        \item Milk \item Eggs \item Potatos
%      \end{itemize}
%
%    \column{0.33\textwidth}
%      Enumerations
%      \begin{enumerate}
%        \item First, \item Second and \item Last.
%      \end{enumerate}
%
%    \column{0.33\textwidth}
%      Descriptions
%      \begin{description}
%        \item[PowerPoint] Meeh. \item[Beamer] Yeeeha.
%      \end{description}
%  \end{columns}
%\end{frame}
%
%\begin{frame}{Animation}
%  \begin{itemize}[<+- | alert@+>]
%    \item \alert<4>{This is\only<4>{ really} important}
%    \item Now this
%    \item And now this
%  \end{itemize}
%\end{frame}

%\begin{frame}{Figures}
%  \begin{figure}
%    \newcounter{density}
%    \setcounter{density}{20}
%    \begin{tikzpicture}
%      \def\couleur{alerted text.fg}
%      \path[coordinate] (0,0)  coordinate(A)
%                  ++( 90:5cm) coordinate(B)
%                  ++(0:5cm) coordinate(C)
%                  ++(-90:5cm) coordinate(D);
%      \draw[fill=\couleur!\thedensity] (A) -- (B) -- (C) --(D) -- cycle;
%      \foreach \x in {1,...,40}{%
%          \pgfmathsetcounter{density}{\thedensity+20}
%          \setcounter{density}{\thedensity}
%          \path[coordinate] coordinate(X) at (A){};
%          \path[coordinate] (A) -- (B) coordinate[pos=.10](A)
%                              -- (C) coordinate[pos=.10](B)
%                              -- (D) coordinate[pos=.10](C)
%                              -- (X) coordinate[pos=.10](D);
%          \draw[fill=\couleur!\thedensity] (A)--(B)--(C)-- (D) -- cycle;
%      }
%    \end{tikzpicture}
%    \caption{Rotated square from
%    \href{http://www.texample.net/tikz/examples/rotated-polygons/}{texample.net}.}
%  \end{figure}
%\end{frame}

%\begin{frame}{Tables}
%  \begin{table}
%    \caption{Largest cities in the world (source: Wikipedia)}
%    \begin{tabular}{lr}
%      \toprule
%      City & Population\\
%      \midrule
%      Mexico City & 20,116,842\\
%      Shanghai & 19,210,000\\
%      Peking & 15,796,450\\
%      Istanbul & 14,160,467\\
%      \bottomrule
%    \end{tabular}
%  \end{table}
%\end{frame}

%\begin{frame}{Blocks}
%  Three different block environments are pre-defined and may be styled with an
%  optional background color.
%
%  \begin{columns}[T,onlytextwidth]
%    \column{0.5\textwidth}
%      \begin{block}{Default}
%        Block content.
%      \end{block}
%
%      \begin{alertblock}{Alert}
%        Block content.
%      \end{alertblock}
%
%      \begin{exampleblock}{Example}
%        Block content.
%      \end{exampleblock}
%
%    \column{0.5\textwidth}
%
%      \metroset{block=fill}
%
%      \begin{block}{Default}
%        Block content.
%      \end{block}
%
%      \begin{alertblock}{Alert}
%        Block content.
%      \end{alertblock}
%
%      \begin{exampleblock}{Example}
%        Block content.
%      \end{exampleblock}
%
%  \end{columns}
%\end{frame}

%\begin{frame}{Math}
%  \begin{equation*}
%    e = \lim_{n\to \infty} \left(1 + \frac{1}{n}\right)^n
%  \end{equation*}
%\end{frame}

%\begin{frame}{Line plots}
%  \begin{figure}
%    \begin{tikzpicture}
%      \begin{axis}[
%        mlineplot,
%        width=0.9\textwidth,
%        height=6cm,
%      ]
%
%        \addplot {sin(deg(x))};
%        \addplot+[samples=100] {sin(deg(2*x))};
%
%      \end{axis}
%    \end{tikzpicture}
%  \end{figure}
%\end{frame}

%\begin{frame}{Bar charts}
%  \begin{figure}
%    \begin{tikzpicture}
%      \begin{axis}[
%        mbarplot,
%        xlabel={Foo},
%        ylabel={Bar},
%        width=0.9\textwidth,
%        height=6cm,
%      ]
%
%      \addplot plot coordinates {(1, 20) (2, 25) (3, 22.4) (4, 12.4)};
%      \addplot plot coordinates {(1, 18) (2, 24) (3, 23.5) (4, 13.2)};
%      \addplot plot coordinates {(1, 10) (2, 19) (3, 25) (4, 15.2)};
%
%      \legend{lorem, ipsum, dolor}
%
%      \end{axis}
%    \end{tikzpicture}
%  \end{figure}
%\end{frame}

%\begin{frame}{Quotes}
%  \begin{quote}
%    Veni, Vidi, Vici
%  \end{quote}
%\end{frame}

%{%
%\setbeamertemplate{frame footer}{My custom footer}
%\begin{frame}[fragile]{Frame footer}
%    \themename defines a custom beamer template to add a text to the footer. It can be set via
%    \begin{verbatim}\setbeamertemplate{frame footer}{My custom footer}\end{verbatim}
%\end{frame}
%}

%\begin{frame}{References}
%  Some references to showcase [allowframebreaks] \cite{knuth92,ConcreteMath,Simpson,Er01,greenwade93}
%\end{frame}
%
%\section{Conclusion}

%\begin{frame}{Summary}
%
%  Get the source of this theme and the demo presentation from
%
%  \begin{center}\url{github.com/matze/mtheme}\end{center}
%
%  The theme \emph{itself} is licensed under a
%  \href{http://creativecommons.org/licenses/by-sa/4.0/}{Creative Commons
%  Attribution-ShareAlike 4.0 International License}.
%
%%  \begin{center}\ccbysa\end{center}
%
%\end{frame}

%{\setbeamercolor{palette primary}{fg=black, bg=yellow}
%\begin{frame}[standout]
%  Questions?
%\end{frame}
%}

\appendix

%\begin{frame}[fragile]{Backup slides}
%  Sometimes, it is useful to add slides at the end of your presentation to
%  refer to during audience questions.
%
%  The best way to do this is to include the \verb|appendixnumberbeamer|
%  package in your preamble and call \verb|\appendix| before your backup slides.
%
%  \themename will automatically turn off slide numbering and progress bars for
%  slides in the appendix.
%\end{frame}

%\begin{frame}[allowframebreaks]{References}
%
%  \bibliography{demo}
%  \bibliographystyle{abbrv}
%
%\end{frame}

\end{document}
